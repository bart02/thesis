\chapter{Conclusion}
\label{chap:conclusion}

This thesis has explored the efficacy of knowledge distillation techniques to enhance the performance of sequence classification models by leveraging the capabilities of LLMs. The central aim was to investigate the process of distilling an LLM into a text classification model, to develop a methodological framework that improves the efficiency and effectiveness of smaller models, and to evaluate the efficacy of this methodology.

The thesis has made several significant contributions to the field of NLP and knowledge distillation:
\begin{itemize}
    \item The thesis has provided a comprehensive overview of knowledge distillation techniques in ML models, focusing on recent advancements in the field of LLMs. The study has explored the application of KD methods to optimize LLMs, making them more practical and sustainable for real-world applications.
    \item The study introduced a novel distillation pipeline that combines the generative capabilities of the T5 model with the classification efficiency of the RoBERTa model. This pipeline utilizes the strengths of each model to improve the performance of sequence classification tasks, as evidenced by the experimental results on the Web of Science dataset.
    \item The experiments conducted as part of this thesis provided empirical evidence of the potential benefits of integrating generated by LLM rationales into the training process.
\end{itemize}

The findings from this research have practical implications, suggesting that knowledge distillation can empower smaller models by transferring knowledge from LLMs. The proposed pipeline, when applied, demonstrated a tangible improvement in classification accuracy on the WOS dataset, thereby validating the effectiveness of the distillation approach in real-world scenarios.

In conclusion, this thesis has convincingly demonstrated that knowledge distillation holds promise in enhancing the performance of smaller NLP models. By effectively leveraging the distilled knowledge from LLMs, significant improvements in model efficiency and accuracy can be achieved. However, the variable results across different datasets underscore the complexity of the task and the pressing need for further research in this evolving field.